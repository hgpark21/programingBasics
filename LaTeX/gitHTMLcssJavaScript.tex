%\documentclass{tufte-handout}

%\geometry{showframe} % display margins for debugging page layout

\documentclass{amsart}

%\title{An Example of the Usage of the Tufte-Handout Style\thanks{Inspired by Edward~R. Tufte!}}
%\author[The Tufte-LaTeX Developers]{The Tufte-\LaTeX\ Developers}

\usepackage{amsmath,amssymb,amsthm}
\usepackage{fullpage}
\usepackage{color}
\usepackage{enumitem}
\usepackage{tabularx,graphicx}

%  \setkeys{Gin}{width=\linewidth,totalheight=\textheight,keepaspectratio}
%  \graphicspath{{graphics/}} % set of paths to search for images

\usepackage{booktabs}
\usepackage{units}
\usepackage{multicol}
\usepackage{fancyvrb} 
  \fvset{fontsize=\normalsize}

\usepackage{lipsum}   % filler text

\newcommand{\commandname}[1]{\textcolor{blue}{$\mbox{\ttfamily#1}$}}
\newcommand{\anglename}[1]{$<\!\mbox{\ttfamily#1}\!>$}
\newcommand{\gitcommand}[1]{\textcolor{blue}{{\ttfamily git} \commandname{#1}}}

\newcommand{\git}{\textcolor{red}{{\ttfamily git}}}
\newcommand{\github}{\textcolor{red}{{\ttfamily GitHub}}}
\newcommand{\html}{\textcolor{red}{{\ttfamily html}}}
\newcommand{\css}{\textcolor{blue}{{\ttfamily CSS}}}
\newcommand{\javascript}{\textcolor{blue}{{\ttfamily JavaScript}}}
\newcommand{\java}{\textcolor{blue}{{\ttfamily Java}}}

\newcommand{\newpara}[1]{\par\vspace{0.5em}\paragraph{\scshape#1}}

\newcommand{\afakeline}{\textcolor{white}{a fake line here}}

\title{\github/\git, \html, \css, and \javascript}
\author{Heunggi Park}
\date{June 13, 2020}

\begin{document}

\maketitle
%%% header %%%
%\begin{center}
%\textbf{\large \git, \html, \css, and \javascript} \\[1em]% title
%\textsc{Heunggi Park} \\[0.5em]
%\textsc{June 13, 2020}
%\end{center}

\begin{abstract}
	Here is an abstract of the article.
\end{abstract}

\section{\github/\git\ commands}

\subsection{What is \github?}\afakeline\par\vspace{0.5em}\noindent

\newpara{Write Better Code}
On \github, lightweight code review tools are built into every pull request. Your team can create review processes that improve the quality of your code and fit neatly into your workflow.

\newpara{Start with a Pull Request}
Pull requests are fundamental to how teams review and improve code on GitHub. Evolve projects, propose new features, and discuss implementation details before changing your source code.

\newpara{Make a Change}
Start a new feature or propose a change to existing code with a pull request—a base for your team to coordinate details and refine your changes.


\subsection{List of Basic \git\ Commands}\afakeline%\par\vspace{0.5em}\noindent

\begin{multicols}{2}

\begin{description}[itemsep=0.25em]
\item[\gitcommand{clone \anglename{url}}]
\item[\gitcommand{add \anglename{filename}}]
\item[\gitcommand{commit -m "message"}]
\item[\gitcommand{commit -am "message"}]
\item[\gitcommand{push}] (\gitcommand{push "--set-upstream"})
\item[\gitcommand{pull}] (Cf. \gitcommand{fetch})
\item[\gitcommand{statue}]
\item[\gitcommand{log}]
\item[\gitcommand{branch}] (\gitcommand{branch \anglename{branchname}})
\item[\gitcommand{checkout \anglename{branchname}}]
\item[\gitcommand{merge \anglename{branchname}}]
\end{description}
\end{multicols}


\subsection*{\gitcommand{pull} vs \gitcommand{fetch}} \afakeline\par\vspace{0.5em}\noindent
The command \gitcommand{fetch} tells your local \git\ to retrieve the latest {\em meta-data info\/} from the original (yet doesn't do any file transferring. It's more like just checking to see if there are any changes available), whereas \gitcommand{pull} {\bf does that AND} brings those changes from the remote repository.

\section{\html\ 5}

\subsection{The \html\ Elements}
\subsection{Forms}

\section{\css}

\subsection{\css\ Selectors}

\subsection{Responsive Design}

\begin{itemize}
	\item viewport
	\item Media Queries
	\item Flexbox
	\item Grids
\end{itemize}


\section{\javascript}
\subsection{Basics}
\subsection{\textcolor{blue}{\texttt Node.js}}
\subsection{\textcolor{blue}{\texttt AJAX}}

\end{document}